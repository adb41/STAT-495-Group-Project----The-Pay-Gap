\documentclass[12pt]{article}
\usepackage[utf8]{inputenc}
\usepackage{graphicx}

\begin{document}

\begin{titlepage}
   \begin{center}
       \vspace*{1cm}

       \textbf{STAT 495 Group Project}

       \vspace{0.5cm}
        The Pay Gap
            
       \vspace{1cm}

       \textbf{Alex Dolan Balin, Erdene Enkhbadral, Anthony Martinez}

       \vfill
        
            

     
       \includegraphics{university.png}
       
              \vspace{1.5cm}
            
       Mathematics \& Statistics\\
       California State University Long Beach\\
       USA\\
       May 2021
            
   \end{center}
\end{titlepage}

\section{Introduction}

The “pay gap” is a colloquial term which refers to the position that on average women are paid less than men. Its existence is contested in public spaces, wherein subscribers and skeptics each claim their opponents misappropriate statistical methods to support their position. To determine for ourselves the validity of these claims, our team found a dataset to test the claims. Hosted on Kaggle.com, the dataset was published by the online company Glassdoor.com, which aims to enable employees from different companies to review their employers and seek better employment. Comprised of one thousand randomly selected use profiles, the dataset represents various details of each person, namely their sex (female/male), their base income, as well as any bonus pay they may earn annually. With this information, our team endeavored to determine the answer to two questions: is there a statistically significant difference in the total annual income between people who report to be female and people who report to be male, and is that difference (or lack thereof) consistent along the other variables in the dataset?

\section{Analysis}

To address both questions, our report will visualize differences in base income, bonus pay, and their sum, particularly when grouped by the other variables available in the dataset. To demonstrate statistically significant differences in their total annual income in all the various subgroups, we will construct and visualize confidence intervals on bootstrapped data.  

\subsection{Exploratory Data Analysis}

\end{document}
